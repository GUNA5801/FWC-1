\def\mytitle{ASSIGNMENT}
\def\myauthor{GUNA VARDHAN}
\def\contact{gunavardhan.nagamalla@gmail.com}
\def\mymodule{Future Wireless Communications (FWC)}
\documentclass[journal,12pt,two column]{IEEEtran}

\usepackage{setspace}
\usepackage{gensymb}
\usepackage{xcolor}
\usepackage{caption}
\usepackage[hyphens,spaces,obeyspaces]{url}
\usepackage[cmex10]{amsmath}
\usepackage{mathtools}
\singlespacing
\usepackage{amsthm}
\usepackage{mathrsfs}
\usepackage{txfonts}
\usepackage{stfloats}
\usepackage{cite}
\usepackage{cases}
\usepackage{subfig}
\usepackage{longtable}
\usepackage{multirow}
\twocolumn


\usepackage{graphicx}
\graphicspath{{./images/}}
\usepackage[colorlinks,linkcolor={black},citecolor={blue!80!black},urlcolor={blue!80!black}]{hyperref}
\usepackage[parfill]{parskip}
\usepackage{lmodern}
\usepackage{tikz}
\usepackage{circuitikz}
\usepackage{karnaugh-map}
\usepackage{pgf}
\usepackage[hyphenbreaks]{breakurl}

\usepackage{tabularx}
\usetikzlibrary{calc}

\renewcommand*\familydefault{\sfdefault}
\usepackage{watermark}
\usepackage{lipsum}
\usepackage{xcolor}
\usepackage{listings}
\usepackage{float}
\usepackage{titlesec}
\DeclareMathOperator*{\Res}{Res}
%\renewcommand{\baselinestretch}{2}
\renewcommand\thesection{\arabic{section}}
\renewcommand\thesubsection{\thesection.\arabic{subsection}}
\renewcommand\thesubsubsection{\thesubsection.\arabic{subsubsection}}

\renewcommand\thesectiondis{\arabic{section}}
\renewcommand\thesubsectiondis{\thesectiondis.\arabic{subsection}}
\renewcommand\thesubsubsectiondis{\thesubsectiondis.\arabic{subsubsection}}

% correct bad hyphenation here
\hyphenation{op-tical net-works semi-conduc-tor}

\titlespacing{\subsection}{1pt}{\parskip}{3pt}
\titlespacing{\subsubsection}{0pt}{\parskip}{-\parskip}
\titlespacing{\paragraph}{0pt}{\parskip}{\parskip}
\newcommand{\figuremacro}[5]{
    \begin{figure}[#1]
        \centering
        \includegraphics[width=#5\columnwidth]{#2}
        \caption[#3]{\textbf{#3}#4}
        \label{fig:#2}
    \end{figure}
}

\lstset{
frame=single, 
breaklines=true,
columns=fullflexible
}

%\thiswatermark{\centering \put(400,-128.0){\includegraphics[scale=0.3]{logo}} }
\title{\mytitle}
\author{\myauthor\hspace{1em}\\\contact\\IITH\hspace{0.5em}-\hspace{0.6em}\mymodule}
\date{20-12-2022}
\def\inputGnumericTable{}                                 %%
\lstset{
%language=C,
frame=single, 
breaklines=true,
columns=fullflexible
}
 

\begin{document}
%

\theoremstyle{definition}
\newtheorem{theorem}{Theorem}[section]
\newtheorem{problem}{Problem}
\newtheorem{proposition}{Proposition}[section]
\newtheorem{lemma}{Lemma}[section]
\newtheorem{corollary}[theorem]{Corollary}
\newtheorem{example}{Example}[section]
\newtheorem{definition}{Definition}[section]
%\newtheorem{algorithm}{Algorithm}[section]
%\newtheorem{cor}{Corollary}
\newcommand{\BEQA}{\begin{eqnarray}}
\newcommand{\EEQA}{\end{eqnarray}}
\newcommand{\define}{\stackrel{\triangle}{=}}
\bibliographystyle{IEEEtran}

\vspace{3cm}
  \maketitle
  \tableofcontents
 
\section{Question}  
     Q.39. The state diagram of a sequence detector is shown below. State S0 is the initial state of the sequence detector. If the output is 1, then 
     \begin{enumerate}
         \item The sequence 01010 is detected
         \item The sequence 01011 is detected
         \item The sequence 01110 is detected
         \item The sequence 01001is detected
     \end{enumerate}

\begin{tikzpicture} [node distance=2cm]
\centering 
\node[circle, draw, state, initial] (S0) {$S_0$};
\node[circle, draw, state, accepting, right of=S0] (S1) {$S_1$};
\node[circle, draw, state, accepting, right of=S1] (S2) {$S_2$};
\node[circle, draw, state, accepting, right of=S2] (S3) {$S_3$}; 
\node[circle, draw, state, below right of = S1] (S4) {$S_4$};
\path[->] (S0) edge[above] node{0/0} (S1) 
          (S0) edge[loop above] node{1/0} (S0)
          (S1) edge[above] node{1/0} (S2)
          (S1) edge[loop above] node{0/0} (S1)
          (S2) edge[above] node{0/0} (S3) 
          (S2) edge[below,bend left] node{1/0} (S0)
          (S3) edge[below] node{1/0} (S4) 
          (S3) edge[above,bend right] node{0/0} (S1) 
          (S4) edge[below,bend left] node{1/0} (S0)   
          (S4) edge[below,bend right] node{0/1} (S3); 
\end{tikzpicture}
  \section{Components}
  \begin{tabularx}{0.4\textwidth} { 
  | >{\centering\arraybackslash}X 
  | >{\centering\arraybackslash}X 
  | >{\centering\arraybackslash}X
  | >{\centering\arraybackslash}X | }
\hline
 \textbf{Component}& \textbf{Values} & \textbf{Quantity}\\
\hline
Arduino & UNO & 1 \\  
\hline
JumperWires& M-M & 10 \\ 
\hline
Breadboard &  & 1 \\
\hline
LED & &2 \\
\hline
Resistor &220ohms & 1\\
\hline
\end{tabularx}
\begin{center}
Figure.a
\end{center}
\section{State Transition Table}
 \begin{table}[ht]
    \centering
    \caption{State Transition Table}
    \label{tab:state-transition}
    \begin{tabular}{|c|c|c|c|}
        \hline
        \textbf{Current State} & \textbf{Input (x)} & \textbf{Next State} & \textbf{Output (z)} \\ \hline
        S0 & 0 & S0 & 0 \\ \hline
        S0 & 1 & S1 & 0 \\ \hline
        S1 & 0 & S0 & 0 \\ \hline
        S1 & 1 & S2 & 0 \\ \hline
        S2 & 0 & S0 & 0 \\ \hline
        S2 & 1 & S3 & 1 \\ \hline
        S3 & 0 & S0 & 0 \\ \hline
        S3 & 1 & S3 & 1 \\ \hline
    \end{tabular}
\end{table}
\begin{center}
 Truth table Boolean Function "F"
\end{center}
    
\section{Implementation}
  \begin{tabularx}{0.46\textwidth} { 
  | >{\centering\arraybackslash}X 
  | >{\centering\arraybackslash}X 
  | >{\centering\arraybackslash}X  | }
\hline
\textbf{Arduino PIN} & \textbf{INPUT} & \textbf{OUTPUT} \\ 
\hline
\textbf 2 & X & \\
\hline
\textbf 3 & Y & \\
\hline
\textbf 4 && Z \\
\hline
\end{tabularx}
\begin{center}
    Connections
\end{center}
    \paragraph{Procedure}
    
    1. Connect the circuit as per the above table.\\
    2. Connect the output pin to LED\\
    3. Connect inputs to Vcc for logic 1, ground for logic 0\\
    4. Execute the circuit using the below code.\\
   
\begin{tabularx}{0.6\textwidth} { 
  | >{\centering\arraybackslash}X |}
  \hline
  https://github.com/GUNA5801/FWC/blob/main/AVR-GCC/CODE/main.c\\
  \hline
\end{tabularx}
   
5. Change the values of X,Y,Z in the code and verify the Truth Table\\
\bibliographystyle{ieeetr}
\end{document}
